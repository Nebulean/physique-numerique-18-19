% debut d'un fichier latex standard
\documentclass[a4paper,12pt,twoside]{article}

% pour l'inclusion de figures en eps,pdf,jpg
\usepackage{graphicx}
\usepackage{subfigure}
\usepackage{wrapfig}
% quelques symboles mathematiques en plus
\usepackage{amsmath}
% le tout en langue francaise
%\usepackage[french]{babel}
% on peut ecrire directement les caracteres avec l'accent
% a utiliser sur Linux/Windows
\usepackage[utf8]{inputenc}
\usepackage[T1]{fontenc}
% a utiliser sur le Mac
%\usepackage[applemac]{inputenc}
% pour l'inclusion de links dans le document
\usepackage[colorlinks,bookmarks=false,linkcolor=blue,urlcolor=blue]{hyperref}
\usepackage{siunitx}

\paperheight=297mm
\paperwidth=210mm

\setlength{\textheight}{235mm}
\setlength{\topmargin}{-1.2cm} % pour centrer la page verticalement
%\setlength{\footskip}{5mm}
\setlength{\textwidth}{15cm}
\setlength{\oddsidemargin}{0.56cm}
\setlength{\evensidemargin}{0.56cm}

\pagestyle{plain}

% quelques abreviations utiles
\def \be {\begin{equation}}
\def \ee {\end{equation}}
\def \dd  {{\rm d}}

\newcommand{\mail}[1]{{\href{mailto:#1}{#1}}}
\newcommand{\ftplink}[1]{{\href{ftp://#1}{#1}}}
%
% latex SqueletteRapport.tex      % compile la source LaTeX
% xdvi SqueletteRapport.dvi &     % visualise le resultat
% dvips -t a4 -o SqueletteRapport.ps SqueletteRapport % produit un PostScript
% ps2pdf SqueletteRapport.ps      % convertit en pdf

% pdflatex SqueletteRapport.pdf    % compile et produit un pdf

% ======= Le document commence ici ======

\begin{document}
% Le titre, l'auteur et la date
\title{Gravitational systems.\\{\small Apollo 13, atmosphere re-entry and Trojan asteroids \\ Physique Numérique I}\\{\small Rapport 4}}
\date{\today}
\author{Delphine Martres et Damien Korber\\{\small \mail{delphine.martres@epfl.ch} et \mail{damien.korber@epfl.ch}}}
\maketitle
\tableofcontents % Table des matieres

% Quelques options pour les espacements entre lignes, l'identation
% des nouveaux paragraphes, et l'espacement entre paragraphes
\baselineskip=16pt
\parindent=15pt
\parskip=5pt
\newpage


%%%% ON COMMENCE A ECRIRE D'ICI

\section{Introduction}

\section{"Houston, we've had a problem"}
Apollo 13 is located at a distance $r_0 = \SI{314159}{\kilo\meter}$ from the center of Earth.
Its velocity with respect to earth's frame of reference is $v_0 = \SI{1.2}{\kilo\meter\per\second}$.
Because of a fuel shortage, the norm of speed can not be changed.
However, it is possible to ajust its direction.
To bring back Apollo 13's crew, their trajectory needs to at get as close as $h=\SI{10}{\kilo\meter}$ to earth, the distance where the atmosphere slows the ship to break its orbit.
In this experiment, the Moon will be ignored, so it will be a 1 body gravitationnal problem.
%TODO : Est-ce qu'il faut mettre là toutes les constantes ? Y'en a plein et je les définis quand on en a besoin...

\subsection{Initial velocity of Apollo 13}
The goal of this section is to find the componants of the velocity of the probs, so that it get close to earth, at \SI{10}{\kilo\meter}.
Two physical laws are considered to solve this problem. %TODO : C'est pas des lois, t'as un autre mot ?
\begin{itemize}
  \item Conservation of mechanical energy
  \item Conservation of angular momentum
\end{itemize}

\subsubsection{Conservation of mechanical energy}
The mechanical energy of the probe at any moment is given by equation \eqref{eq:1c-em-apollo13}.
\begin{equation}
  E_m = \frac{1}{2}mv^2 - G\frac{mM}{r}
  \label{eq:1c-em-apollo13}
\end{equation}
where $m=\SI{5809}{\kilo\gram}$ is the mass of the probe, $M=\SI{5.972d24}{\kilo\gram}$ is the mass of earth, $G=\SI{6.674d-11}{\cubic\meter\per\kilo\gram\per\square\second}$ is the gravitational constant, $v$ is the speed of the probe with respect to earth's frame of reference, and $r$ is the distance between the probe and the center of earth.\\

By conservation of mechanical energy, two moments are chosen.
\begin{itemize}
  \item $t=\SI{0}{\s}: E_{m,t_0} = \frac{1}{2}mv_0^2 - G\frac{mM}{r_0}$
  \item $t=t_{end}: E_{m,t_{end}} = \frac{1}{2}mv_{end}^2 - G\frac{mM}{h}$
\end{itemize}

Thus, solving $E_{m,t_0} = E_{m,t_{end}}$, the velocity of the probe when it is at a distance $h=\SI{10}{\kilo\meter}$ of earth can be found, and is given by equation \eqref{eq:1c-vitesse-finale}.

\begin{equation}
  v_{end} = \sqrt{v_0^2 - \frac{GM}{2}\left(\frac{1}{h} - \frac{1}{r_0}\right)}
  \label{eq:1c-vitesse-finale}
\end{equation}

\subsubsection{Conservation of angular momentum}
%TODO : À écrire.

\section{"Houston, we're going to have a real problem"}

\section{Two bodies: Earth and Moon}

\section{Three bodies: Earth, Moon and Apollo 13}

\section{Three bodies: Earth, Moon and a third body - Lagrangian points}






\end{document}
