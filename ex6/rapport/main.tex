% debut d'un fichier latex standard
\documentclass[a4paper,12pt,twoside]{article}

% pour l'inclusion de figures en eps,pdf,jpg
\usepackage{graphicx}
\usepackage{subcaption}
\usepackage{wrapfig}
% quelques symboles mathematiques en plus
\usepackage{amsmath}
\usepackage{amsthm} % Pour les preuves
% le tout en langue francaise
%\usepackage[french]{babel}
% on peut ecrire directement les caracteres avec l'accent
% a utiliser sur Linux/Windows
\usepackage[utf8]{inputenc}
\usepackage[T1]{fontenc}
% a utiliser sur le Mac
%\usepackage[applemac]{inputenc}
% pour l'inclusion de links dans le document
\usepackage[colorlinks,bookmarks=false,linkcolor=blue,urlcolor=blue]{hyperref}
\usepackage{siunitx}
% pour les degrés
\usepackage{textcomp}
\paperheight=297mm
\paperwidth=210mm

\setlength{\textheight}{235mm}
\setlength{\topmargin}{-1.2cm} % pour centrer la page verticalement
%\setlength{\footskip}{5mm}
\setlength{\textwidth}{15cm}
\setlength{\oddsidemargin}{0.56cm}
\setlength{\evensidemargin}{0.56cm}

\pagestyle{plain}

% quelques abreviations utiles
\def \be {\begin{equation}}
\def \ee {\end{equation}}
\def \dd  {{\rm d}}

\newcommand{\mail}[1]{{\href{mailto:#1}{#1}}}
\newcommand{\ftplink}[1]{{\href{ftp://#1}{#1}}}

\newcommand{\illabel}[1]{ ~ \refstepcounter{equation}(\theequation)\label{#1}} % Ecrit une équation dans le texte, numérotés.
\newcommand{\mbf}[1]{\mathbf{#1}} % bold font in math
\newcommand{\grad}[1]{\nabla#1}
\newcommand{\Div}[1]{\nabla\cdot\mathbf{#1}}
\newcommand{\rot}[1]{\nable\cross\mathbf{#1}}
\newcommand{\bracket}[1]{\left(#1\right)}
\newcommand{\lapl}[1]{\Delta#1}

%
% latex SqueletteRapport.tex      % compile la source LaTeX
% xdvi SqueletteRapport.dvi &     % visualise le resultat
% dvips -t a4 -o SqueletteRapport.ps SqueletteRapport % produit un PostScript
% ps2pdf SqueletteRapport.ps      % convertit en pdf

% pdflatex SqueletteRapport.pdf    % compile et produit un pdf

% ======= Le document commence ici ======

\begin{document}
% Le titre, l'auteur et la date
\title{Electrostatique\\{\small Physique Numérique I}\\{\small Rapport 6}}
\date{\today}
\author{Delphine Martres et Damien Korber\\{\small \mail{delphine.martres@epfl.ch} et \mail{damien.korber@epfl.ch}}}

\maketitle
\tableofcontents % Table des matieres


% Quelques options pour les espacements entre lignes, l'identation
% des nouveaux paragraphes, et l'espacement entre paragraphes
\baselineskip=16pt
\parindent=15pt
\parskip=5pt
\newpage

%%%% ON COMMENCE A ECRIRE D'ICI

\section{Introduction} %TODO: Virer cette intro

\section{Differential equation and variationnal form}
  \subsection{Establishment of the differential equation for the potential $\phi(\vec{x})$}
    This section will derive the equation that describes the problem.

    \begin{align*}
      \intertext{First, one of Maxwell's equation is used}
      \nabla\cdot\mathbf{D} &= \rho\bracket{\mathbf{x}} \\
      \intertext{where $\mathbf{D}=\epsilon_0\epsilon_r\mathbf{E}$ is the electric displacement field.}
      \nabla\cdot\bracket{\epsilon_0\epsilon_r\mathbf{E}} &= \rho\bracket{\mathbf{x}} \\
      \nabla\cdot\bracket{\epsilon_r\mathbf{E}} &= \frac{\rho\bracket{\mathbf{x}}}{\epsilon_0} \\
      \epsilon_r\Div{E} + \mathbf{E}\cdot\nabla\epsilon_r &= \frac{\rho\bracket{\mbf{x}}}{\epsilon_0} \\
      \intertext{We know that $\mbf{E}$ derive from a potential $\phi$ such that $\mathbf{E}=-\grad{\phi}$. Thus}
      -\epsilon_r\lapl{\phi}-\grad{\phi}\grad{\epsilon_r} &= \frac{\rho\bracket{\mbf{x}}}{\epsilon_0}
      \intertext{This leads to equation \eqref{eq:equa-diff}.}
    \end{align*}
    \begin{equation}
      \centering
      \boxed{\grad{\phi}\grad{\epsilon_r} + \epsilon_r\lapl{\phi} = -\frac{\rho\bracket{\mbf{x}}}{\epsilon_0}}
      \label{eq:equa-diff}
    \end{equation}

  \subsection{Variational form}
    To numerically solve equation \eqref{eq:equa-diff}, the variational form of the problem must be found.

    \begin{align*}
      \intertext{Let $\eta\bracket{\mbf{x}}\subset\mathcal{C}^1\bracket{\Omega}$, $\eta\bracket{\mbf{x}}=0~ \forall \mbf{x}\in\partial\Omega$. Multiply both sides of equation \eqref{eq:equa-diff} by $\eta\bracket{\mbf{x}}$, and integrate over the volume $\Omega$.}
      \int_\Omega \eta\bracket{\mbf{x}}\grad{\phi}\grad{\epsilon_r}d\mbf{x} + \int_\Omega \eta\bracket{\mbf{x}}\epsilon_r\lapl{\phi}d\mbf{x} &= -\int_\Omega \eta\bracket{\mbf{x}}\frac{\rho\bracket{\mbf{x}}}{\epsilon_0}d\mbf{x}
      \intertext{By using formula $\nabla\cdot\bracket{f\nabla g} = \grad{f}\cdot\grad{g} + f\lapl{g}$ where $f=\eta\epsilon_r$ et $g=\phi$:}
      \nabla\cdot\bracket{\eta\epsilon_r\grad{\phi}} &= \grad{\bracket{\eta\epsilon_r}}\cdot\grad{\phi} + \eta\epsilon_r\lapl{\phi} \\
      &= \epsilon_r\grad{\eta}\cdot\grad{\phi} + \eta\grad{\epsilon_r}\cdot\grad{\phi} + \eta\epsilon_r\lapl{\phi} \\
      \Rightarrow \int_\Omega\bracket{\eta\bracket{\mbf{x}}\grad{\phi}\cdot\grad{\epsilon_r} + \eta\bracket{\mbf{x}}\epsilon_r\lapl{\phi}}d\mbf{x} &= \int_\Omega\bracket{\underbrace{\nabla\cdot\bracket{\eta\epsilon_r\grad{\phi}}}_* - \epsilon_r\grad{\eta}\cdot\grad{\phi}}d\mbf{x}
      \intertext{Using Gauss theorem: $*= \int_{\partial\Omega}\eta\epsilon_r\grad{\phi}\cdot d\mbf{\sigma} = 0$ because $\eta\bracket{\mbf{x}}=0~\forall\mbf{x}\in\partial\Omega$ by definition.}
    \end{align*}
    \begin{equation}
      \centering
      \Rightarrow \boxed{\int_\Omega\epsilon_r\grad{\eta}\cdot\grad{\phi}d\mbf{x} = \int_\Omega\eta\bracket{\mbf{x}}\frac{\rho\bracket{\mbf{x}}}{\epsilon_0}d\mbf{x}}
      \label{eq:variational-form}
    \end{equation}

\end{document}
