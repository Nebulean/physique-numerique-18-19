% debut d'un fichier latex standard
\documentclass[a4paper,12pt,twoside]{article}

\usepackage{lipsum}
\usepackage{empheq}

% pour l'inclusion de figures en eps,pdf,jpg
\usepackage{graphicx}
\usepackage{subcaption}
\usepackage{wrapfig}
% quelques symboles mathematiques en plus
\usepackage{amsmath}
\usepackage{amsthm} % Pour les preuves
% le tout en langue francaise
%\usepackage[french]{babel}
% on peut ecrire directement les caracteres avec l'accent
% a utiliser sur Linux/Windows
\usepackage[utf8]{inputenc}
\usepackage[T1]{fontenc}
% a utiliser sur le Mac
%\usepackage[applemac]{inputenc}
% pour l'inclusion de links dans le document
\usepackage[colorlinks,bookmarks=false,linkcolor=blue,urlcolor=blue]{hyperref}
\usepackage{siunitx}
% pour les degrés
\usepackage{textcomp}
\paperheight=297mm
\paperwidth=210mm

\setlength{\textheight}{235mm}
\setlength{\topmargin}{-1.2cm} % pour centrer la page verticalement
%\setlength{\footskip}{5mm}
\setlength{\textwidth}{15cm}
\setlength{\oddsidemargin}{0.56cm}
\setlength{\evensidemargin}{0.56cm}

\pagestyle{plain}

% quelques abreviations utiles
\def \be {\begin{equation}}
\def \ee {\end{equation}}
\def \dd  {{\rm d}}

\newcommand{\mail}[1]{{\href{mailto:#1}{#1}}}
\newcommand{\ftplink}[1]{{\href{ftp://#1}{#1}}}

\newcommand{\illabel}[1]{ ~ \refstepcounter{equation}(\theequation)\label{#1}} % Ecrit une équation dans le texte, numérotés.
\newcommand{\mbf}[1]{\mathbf{#1}} % bold font in math
\newcommand{\grad}[1]{\nabla#1}
\newcommand{\Div}[1]{\nabla\cdot\mathbf{#1}}
\newcommand{\rot}[1]{\nable\cross\mathbf{#1}}
\newcommand{\bracket}[1]{\left(#1\right)}
\newcommand{\sqbracket}[1]{\left[#1\right]}
\newcommand{\lapl}[1]{\Delta#1}

%
% latex SqueletteRapport.tex      % compile la source LaTeX
% xdvi SqueletteRapport.dvi &     % visualise le resultat
% dvips -t a4 -o SqueletteRapport.ps SqueletteRapport % produit un PostScript
% ps2pdf SqueletteRapport.ps      % convertit en pdf

% pdflatex SqueletteRapport.pdf    % compile et produit un pdf
% \message{================> TAILLE DE LA POLICE EN CM \printinunitsof{cm}\prntlen{\textwidth}}

% ======= Le document commence ici ======

\begin{document}
% Le titre, l'auteur et la date
\title{Wave equation in an inhomogeneous environment\\{\normalsize Normal modes and propagation of the wave of a tsunami}\\{\small Physique Numérique I}\\{\small Rapport 7}}
\date{\today}
\author{Delphine Martres and Damien Korber\\{\small \mail{delphine.martres@epfl.ch} and \mail{damien.korber@epfl.ch}}}

\maketitle
\tableofcontents % Table des matieres


% Quelques options pour les espacements entre lignes, l'identation
% des nouveaux paragraphes, et l'espacement entre paragraphes
\baselineskip=16pt
\parindent=15pt
\parskip=5pt
\newpage

%%%% ON COMMENCE A ECRIRE D'ICI

\section{Introduction}
This report will focus on the study of the behaviour of waves in an unidimensional environment.
These waves will be governed by equations \eqref{eq:ondes-A}, \eqref{eq:ondes-B} and \eqref{eq:ondes-C}.

\begin{align}
  \text{Equation A: }&~\frac{\partial^2f}{\partial t^2}=u^2\frac{\partial^2f}{\partial x^2}\label{eq:ondes-A}\\
  \text{Equation B: }&~\frac{\partial^2f}{\partial t^2}=\frac{\partial}{\partial x}\bracket{u^2\frac{\partial f}{\partial x}}\label{eq:ondes-B}\\
  \text{Equation C: }&~\frac{\partial^2f}{\partial t^2}=\frac{\partial^2}{\partial x^2}\bracket{u^2f}\label{eq:ondes-C}
\end{align}

The domain considered is $x\in\left[0,L\right]$ with different border condition depending on the situation.
The initial condition is given by $f(x,t=0)=0$.

  % \newpage
  % \begin{thebibliography}{99}
  %
  % \end{thebibliography}

\end{document}
